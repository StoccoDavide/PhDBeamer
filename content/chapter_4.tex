%!TEX root = main.tex

\section{Index Reduction Algorithm}

\begin{frame}
  \tableofcontents[currentsection]
\end{frame}

\begin{frame}{Index Reduction Algorithm}{The Basic Concept}
  The index reduction algorithm is based on the following three steps \dots
  \begin{columns}
    \begin{column}[c]{0.2\textwidth}
      \flushright
      \vspace{-2.5em}%
      \begin{tikzpicture}[overlay, remember picture]
        \draw[fg_sl_color, thick, -stealth] (0.5,-1.95) -- (0.0,-1.95) -- (0.0,1.95) -- (0.5,1.95);
        \node[text width=0.75\textwidth, align=center] at (-1.0,0.0) {If $\mE$ is singular};
        %\draw[fg_sl_color, thick, -stealth] (-1.0,-0.5) -- (-1.0,-2.75);
      \end{tikzpicture}
    \end{column}
    \begin{column}[c]{0.85\textwidth}
      \begin{enumerate}
        \item Let us consider a generic system of \acp{DAE} of the form
        \begin{equation*}
          \mF = \mA \, \mxp - \mb = \m{0} \text{.}
        \end{equation*}
        \item We want to separate the system into \textbf{differential} and \textbf{algebraic} equations and express the system in \textbf{semi-explicit} form
        \begin{equation*}
          \left\{\!\!\!\begin{array}{r@{~}c@{~}l}
            \mE \, \mxp &=& \mg \\
            \m{0}\phantom{\prime} &=& \ma
          \end{array}\!\!\!\right. \text{.}
        \end{equation*}
        \item The index of the system is reduced by differentiating the algebraic equations $\ma = \m{0}$.
      \end{enumerate}
    \end{column}
  \end{columns}
  \vspace{0.75em}
  \dots and it can be applied until a \acs{ODE} system with invariants is obtained.
\end{frame}

\begin{frame}{Index Reduction Algorithm}{Separation of Differential and Algebraic Equations}
  \begin{itemize}
    \item Let us consider a generic system of \acp{DAE} of the form
    \begin{equation*}
      \mF = \mA \, \mxp - \mb = \m{0} \text{.}
    \end{equation*}
    %
    \item The differential equations can be separated from the algebraic ones by exploiting the kernel $\mK$ and its orthogonal complement $\mN$ of $\mE^\top$ such that
    \begin{equation*}
      \left\{\!\!\!\begin{array}{r@{~}c@{~}l}
        \mE \, \mxp &=& \mg \\
        \m{0}\phantom{\prime} &=& \ma
      \end{array}\right.
      \qquad \text{where} \qquad
      \begin{array}{r@{~}c@{~}l}
        \mE &=& \mN \, \mA \text{,} \\
        \mg &=& \mN \, \mb \text{,} \\
        \ma &=& \mK \, \mb \text{.}
      \end{array}
    \end{equation*}

    \begin{bbox}[Kernel Computation]
      To calculate the kernel $\mK$ and its orthogonal complement $\mN$ of $\mE^\top$ we use \ac{LU} or \ac{FFLU} matrix factorizations.
    \end{bbox}
  \end{itemize}
\end{frame}

\begin{frame}{Index Reduction Algorithm}{Differentiation of Algebraic Equations}
  \begin{itemize}
    \item We can differentiate the algebraic equations $\ma$
    \begin{equation*}
      \dfrac{\mathrm{d}}{\mathrm{d}t} \ma = \mAd \, \mxp - \mgd \text{.}
    \end{equation*}
    %
    \item The new system of \acp{DAE} with reduced index is of the form
    %
    \begin{align*}
      \mF = \mA \, \mxp - \mb = \m{0}
      \hspace{0.75em} \text{with} \hspace{0.75em}
      \mA = \begin{bmatrix} \mE \\ \mAd \end{bmatrix}
      \hspace{0.75em} \text{and} \hspace{0.75em}
      \mb = \begin{bmatrix} \mg \\ \mgd \end{bmatrix} \text{.}
    \end{align*}
    %
    \item The differential index of the system has been reduced by one
  \end{itemize}
  %
  \begin{bbox}[A Sequential Algorithm \dots]
    This algorithm is applied repeatedly until any algebraic equation $\mA$ is left, or equivalently until the matrix $\mE$ is non-singular.
  \end{bbox}
\end{frame}

\begin{frame}{Index Reduction Algorithm (3/3)}
  There are dark forces at work \dots
  \begin{enumerate}
    \item \textbf{Expression swell}: during the separation and differentiation of the algebraic part, the size of the symbolic expressions can grow.
    \begin{itemize}
      \item \Maple{} is very sensitive to expression swell.
    \end{itemize}
    %
    \item \textbf{Numerical stability}: symbolic manipulation does not guarantee numerical stability.
    \begin{itemize}
      \item Useless simulation results.
    \end{itemize}
    \end{enumerate}
\end{frame}

% That's all Folks!
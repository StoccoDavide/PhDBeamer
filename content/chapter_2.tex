%!TEX root = main.tex

\section{\aclp{DAE}}

\begin{frame}{\aclp{DAE}}{A Brief Introduction}
  \begin{columns}
    \begin{column}{0.65\textwidth}
      \acp{DAE} are \dots
      \begin{enumerate}
        \item a \textbf{generalization} of algebraic equations and \acp{ODE} \\
      \end{enumerate}
      \centering{\begin{tikzpicture}[scale=0.5]
          \node at (0,0) {%
            $\begin{array}{c}%
            \text{\hi{\acp{DAE}}} \\%
            \mF = \m{0}~\text{with $\m{JF}_{\mxp}$ singular}%
            \end{array}$%
          };
          \draw[fg_sl_color, thick, -stealth] (+1,-1) -- (+2,-2);
          \node at (+3,-3) {%
            $\begin{array}{c}%
              \text{\hi{\acp{ODE}}} \\
              \mxp = \m{f}(\mx, t)
            \end{array}$%
          };
          \draw[fg_sl_color, thick, -stealth] (-1,-1) -- (-2,-2);
          \node at (-3,-3) {%
            $\begin{array}{c}%
              \text{\hi{Algebraic}} \\
                \m{F}(\mx, t) = \m{0}
            \end{array}$%
          };
        \end{tikzpicture}}
        \begin{enumerate}\setcounter{enumi}{1}
        \item equivalent to an \ac{ODE} system with \textbf{constraints}.
        \item \textbf{state-of-the-art} in the modeling of dynamical systems:
        \begin{itemize}
          \item differential equations describe the system's \textbf{dynamics};
          \item algebraic equations constrain the system to a \textbf{manifold}.
        \end{itemize}
      \end{enumerate}
    \end{column}
    \begin{column}{0.35\textwidth}
      \centering
      \small
      \begin{tikzpicture}[scale=0.5]
        \fill (-1.5,0) rectangle (1.5,0.1);
        \draw[tx_sl_color!10] (0,0) -- (-90:4) node [fill, circle]{};
        \draw[tx_sl_color!10] (0,0) -- (-120:4) node [fill, circle]{};
        \draw[fg_sl_color, thick, dashed] (2.8284, -2.8284) arc (-45:-135:4);
        \draw (0,0) -- (-60:4) node[fill, circle](m){};
        \node at (m) [below, yshift=-2mm] {$m$};
        \node at (1.2142, -1.4142) {$\ell$};
      \end{tikzpicture}
      \begin{equation*}
        \begin{array}{c}
          \text{\ac{ODE}} \\ \text{model}
        \end{array}
        \begin{cases}
          \theta^\prime = \omega \\
          \omega^\prime = -\dfrac{g}{\ell} \sin(\theta)
        \end{cases}
      \end{equation*}
      \begin{equation*}
        \begin{array}{c}
          \text{\ac{DAE}} \\ \text{model}
        \end{array}
        \begin{cases}
          x^\prime = u \\
          y^\prime = v \\
          u^\prime = -2x\lambda \\
          v^\prime = -2y\lambda - gm \\
          \, \textcolor{fg_sl_color}{0 = x^2 + y^2 - \ell^2}
        \end{cases}
      \end{equation*}
    \end{column}
  \end{columns}
\end{frame}

\subsection{Solution of \aclp{DAE}}

\begin{frame}{Solution of \aclp{DAE}}{\underline{\acp{DAE} are not \acp{ODE} cit. Linda Petzold}}
  \vspace{-1.5cm}\hspace{4.25cm}\begin{tikzpicture}[scale=0.6]
    \draw[fg_sl_color, thick, -stealth] (0.0,0.0) -- (-1.0,1.0);
    \node at (2.75,0.0) [below]{\small \hi{Someone said that \dots}};
  \end{tikzpicture} \\[1.0em]

  Is the solution of \acp{DAE} a straightforward extension of \acp{ODE}?
  \begin{itemize}
    \item \textbf{No}, otherwise we would not have been here.
    \item Numerical integration of \acp{DAE} is \textbf{challenging}.
    \item A \textbf{reformulation} of the \ac{DAE} system is typically required.
  \end{itemize}
  \vspace{1.0em}
  We end up with \textbf{two} main solution approaches \dots
  \vspace{1.5em}
  \begin{columns}
    \begin{column}[t]{0.5\textwidth}
      \centering
      \hi{\large \fbox{Direct Discretization}}
    \end{column}
    \begin{column}[t]{0.5\textwidth}
      \centering
      \hi{\large \fbox{Index Reduction}}
    \end{column}
  \end{columns}
  \vspace{1.5em}
  \centering{\hi{\large \dots or a mix between the two!}}
\end{frame}

\subsubsection{Direct Discretization}

\begin{frame}{Solution of \aclp{DAE}}{Direct Discretization}
  \begin{itemize}
    \item Approximate $\mxp$ through a \textbf{discretization formula} like a \emph{finite difference scheme} or a \emph{polynomial quadrature}, \eg{}, the Radau collocation.
    \item This discretization can be written as an \textbf{implicit Runge-Kutta} method as
    \begin{equation*}
      \begin{array}{l}
        \m{F}\left(\m{x}_{k} + h_k\displaystyle\sum\nolimits_{j=1}^{s}a_{ij} \m{K}_j, \m{K}_i, t_{k} + c_i h_k\right) = \m{0} \\[0.5em]
        \m{x}_{k+1} = \m{x}_{k} + h_k\displaystyle\sum\nolimits_{i=1}^{s} b_i \m{K}_i
      \end{array}
      \quad \text{for} \quad
      \begin{array}{c}
        i = 1, \dots, s \\
        k = 0, 1, \dots
      \end{array}
    \end{equation*}
    If $(a_{ij})$ is non-singular and $h_k$ small enough, we can solve for $\mxp_k = \m{K}_i$ at each step.
    \item ATTENTION: some order of convergence can be lost depending on the index (defined later)
  \end{itemize}
\end{frame}

\subsubsection{Index Reduction}

\begin{frame}{Solution of \aclp{DAE}}{The Index \dots Or better Yet, The Indices}
  The index is a ``measure of difficulty'' in the analytical or numerical treatment of the \acp{DAE} \dots
  \begin{itemize}
    \item Different approaches in classifying such difficulties led to \textbf{different index concepts}:
    \begin{columns}
      \begin{column}[t]{0.45\textwidth}
        \centering\small
        differential \\
        structural \\
        tractability
      \end{column}
      \begin{column}[t]{0.45\textwidth}
        \centering\small
        geometrical \\
        perturbation \\
        strangeness
      \end{column}
    \end{columns}
    \item Some indices coincide under certain conditions.
    \item The old-but-gold is the \textbf{differential index}, or ``the'' index.
  \end{itemize}
  \vspace{0.5em}
  \begin{bbox}[The Differential Index]
    It is the minimum number of differentiations required to transform a \ac{DAE} system into its underlying \ac{ODE} system.
  \end{bbox}
\end{frame}

\begin{frame}{Solution of \aclp{DAE}}{Index Reduction}
  Index can be reduced by one doing one differentiation and algebraic manipulating the \ac{DAE} system \dots
  \begin{bbox}[The Index Reduction Process]
  \vspace{-1.0em}
  \begin{equation*}
      \begin{array}{c}
        \text{High-index} \\
        \text{\ac{DAE}}
      \end{array}
      \quad \autorightarrow{\text{index}}{\text{reduction}} \quad
      \begin{array}{c}
        \text{\ac{ODE}}
        \text{or} \\
        \text{Index-1 \ac{DAE}}  \\
      \end{array} {\hspace{-0.75em}+\,\text{Invariants}}
      \quad \autorightarrow{\text{numerical}}{\text{integration}} \quad
      \begin{array}{c}
        \text{Solution of} \\
        \text{original \ac{DAE}}
      \end{array}
    \end{equation*}
  \end{bbox}
  \vspace{1.0em}
  Some remarks on index reduction \dots
  \begin{itemize}
    \item it is \textbf{Pre-processing} step prior to numerical integration.
    \item it is carried out through \textbf{symbolic computation} or \textbf{automatic differentiation}.
    \item there is no simple recipe for index reduction exists.
  \end{itemize}
\end{frame}

\begin{frame}{Solution of \aclp{DAE}}{Differential and Algebraic Equations Separation}
  \hic{How do we reduce the index of a \ac{DAE} system?}
  \begin{bbox}[The Fundamental Idea]
    Some index reduction techniques are based on the \textbf{separation} of the differential and algebraic equations.
  \end{bbox}
  \vspace{1.0em}
  Easy to say, not so easy to do \dots
  \vspace{0.5em}
  \begin{columns}
    \centering
    \begin{column}[t]{0.45\textwidth}
      \centering
      \hi{Numerically} \\
      \centering\small
      \textcolor{mycolor5!90!black}{Numerical linear algebra} \\
      \textcolor{mycolor3!90!black}{Automatic differentiation} \\
      \textcolor{mycolor2!90!black}{Track of the separators chain} \\
      \textcolor{mycolor5!90!black}{Numerically intensive}
    \end{column}
    \begin{column}[t]{0.45\textwidth}
      \centering
      \hi{Symbolically} \\
      \centering\small
      \textcolor{mycolor3!90!black}{Symbolic linear algebra} \\
      \textcolor{mycolor5!90!black}{Symbolic differentiation} \\
      \textcolor{mycolor5!90!black}{Just manipulate the equations} \\
      \textcolor{mycolor3!90!black}{Symbolically intensive \emph{pre-processing}}
    \end{column}
  \end{columns}
\end{frame}

\begin{frame}{Solution of \aclp{DAE}}{A New Algorithm for Index Reduction}
  We propose an algorithm for index reduction based on \dots
  \begin{enumerate}
    \item \textbf{none} a priori knowledge of the system structure;
    \item \textbf{basic} use of symbolic computation and linear algebra techniques.
  \end{enumerate}
  \vspace{0.5em}
  \begin{columns}
    \centering
    \begin{column}[t]{0.25\textwidth}
      \centering
      \hi{Factorization} \\
      LU \\ Fraction-Free LU
    \end{column}
    \begin{column}[t]{0.25\textwidth}
      \centering
      \hi{Differentiation} \\[0.25em]
      $\dfrac{\text{d}}{\text{d}\textbf{x}}$ \vspace*{0.15cm}
    \end{column}
  \end{columns}
  \vspace{0.5em}
  The tools we are going to use are \dots
  \begin{enumerate}
    \item \textbf{\Maple{}} for symbolic manipulation;
    \item \textbf{\Matlab{}} for numerical evaluation.
  \end{enumerate}
\end{frame}

% That's all Folks!
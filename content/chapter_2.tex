%!TEX root = main.tex

\section{\aclp{DAE}}

\begin{frame}{\aclp{DAE}}{A Brief Introduction}
  \begin{columns}
    \begin{column}{0.65\textwidth}
      \begin{enumerate}[<+->]
        \item \acsp{DAE} are a \textbf{generalization} of algebraic equations and \acsp{ODE} \\
        \centering{\begin{tikzpicture}[scale=0.5]
        \node at (0,0) {%
          $\begin{array}{c}%
          \text{\hi{\acsp{DAE}}} \\%
          \mF = \m{0}~\text{with $\m{F}_{\mxp}$ singular}%
          \end{array}$%
        };
        \draw[fg_sl_color, thick, -stealth] (-1,-1) -- (-2,-2);
        \node at (-3,-3) {%
          $\begin{array}{c}%
            \text{\hi{Algebraic}} \\
              \m{F}(\mx, t) = \m{0}
          \end{array}$%
        };
        \draw[fg_sl_color, thick, -stealth] (+1,-1) -- (+2,-2);
        \node at (+3,-3) {%
          $\begin{array}{c}%
            \text{\hi{\acsp{ODE}}} \\
            \mxp = \m{f}(\mx, t)
          \end{array}$%
        };
        \end{tikzpicture}}
        \item \acsp{DAE} are composed of
        \begin{itemize}
          \item<2-> differential equations describe the system's \textbf{dynamics}
          \item<2-> algebraic equations constrain the system to a \textbf{manifold} \\
        \end{itemize}
      \end{enumerate}
    \end{column}
    \begin{column}{0.35\textwidth}
      \centering
      \small
      \uncover<2->{\begin{tikzpicture}[scale=0.5]
        \fill (-1.5,0) rectangle (1.5,0.1);
        \draw[tx_sl_color!10] (0,0) -- (-90:4) node [fill, circle]{};
        \draw[tx_sl_color!10] (0,0) -- (-120:4) node [fill, circle]{};
        \draw[fg_sl_color, thick, dashed] (2.8284, -2.8284) arc (-45:-135:4);
        \draw (0,0) -- (-60:4) node[fill, circle](m){};
        \node at (m) [below, yshift=-2mm] {$m$};
        \node at (1.2142, -1.4142) {$\ell$};
      \end{tikzpicture}
      \begin{equation*}
        \begin{array}{c}
          \text{\acs{ODE}} \\ \text{model}
        \end{array}
        \begin{cases}
          \theta^\prime = \omega \\
          \omega^\prime = -\dfrac{g}{\ell} \sin(\theta)
        \end{cases}
      \end{equation*}
      \begin{equation*}
        \begin{array}{c}
          \text{\acs{DAE}} \\ \text{model}
        \end{array}
        \begin{cases}
          x^\prime = u \\
          y^\prime = v \\
          u^\prime = -2x\lambda \\
          v^\prime = -2y\lambda - gm \\
          \, \textcolor{fg_sl_color}{0 = x^2 + y^2 - \ell^2}
        \end{cases}
      \end{equation*}}
    \end{column}
  \end{columns}
\end{frame}

\subsection{Solution of \aclp{DAE}}

\begin{frame}{Solution of \aclp{DAE}}{\acsp{DAE} are not \acsp{ODE}}
  Is the solution of \acsp{DAE} a straightforward extension of \acsp{ODE}?
  \begin{itemize}
    \item<1,3-> Specialized \textbf{numerical techniques} are required
    \item<2-> \textbf{Reformulation} of the \acs{DAE} system is typically required
  \end{itemize}
  \vspace{1.0em}
  \centering
  \begin{tikzpicture}[scale=0.5]
    \uncover<1->{\node at (0,0) {\textbf{Two} main approaches};}
    \visible<2->{\draw[fg_sl_color, thick, -stealth] (-2,-1) -- (-3,-2);}
    \uncover<2->{\node at (-5,-3) {\hi{\large Index reduction}};}
    \visible<2>{\node at (-5,-4) {\hi{\large + \acs{ODE} methods}};}
    \visible<1>{\draw[fg_sl_color, thick, -stealth] (+2,-1) -- (+3,-2);}
    \uncover<1,3->{\node at (+5.75,-3) {\hi{\large Direct discretization}};}
    \visible<3->{
      \node at (0,-6) {\hi{\large or a mix between them!}};
      \draw[fg_sl_color, thick, -stealth] (-1,-3) -- (+1,-3);
      \draw[fg_sl_color, thick, -stealth] (+3,-4) -- (+2,-5);
    }
  \end{tikzpicture} \\[0.5em]
  \uncover<1->{\raggedright\scriptsize{\fullcite{petzold1982differential}}}
\end{frame}

\subsubsection{Direct Discretization}

\begin{frame}{Solution of \aclp{DAE}}{Direct Discretization}
  \begin{itemize}[<+->]
    \item Approximate $\mxp$ through a \textbf{discretization formula} \\
    \begin{small}
      \qquad finite difference scheme \\
      \qquad polynomial quadrature
    \end{small}
    %\item This discretization can be written as an \textbf{implicit Runge-Kutta} method
    \begin{equation*}
      \begin{array}{c}
        \text{Implicit} \\
        \text{Runge-Kutta}
      \end{array}
      \quad
      \begin{array}{l}
        \m{F}\left(\m{x}_{k} + h_k\displaystyle\sum_{j=1}^{s}a_{ij} \m{K}_j, \m{K}_i, t_{k} + c_i h_k\right) = \m{0} \\[0.5em]
        \m{x}_{k+1} = \m{x}_{k} + h_k\displaystyle\sum_{i=1}^{s} b_i \m{K}_i
      \end{array}
      \quad \text{for} \quad
      \begin{array}{l}
        i = 1, \dots, s \\
        k = 0, 1, \dots, m
      \end{array}
    \end{equation*}
    $(a_{ij})$ non-singular + $h_k$ small enough $\rightarrow$ we can solve for $\m{K}_i$ at each $m$-th step
    \item[\textcolor{mycolor3}{\faExclamationTriangle}] Some \textbf{order of convergence} can be lost on some components depending on the \textcolor{mycolor2!95!black}{\textbf{index}}
  \end{itemize}
  \vspace{0.5em}
  \uncover<1->{\scriptsize{\fullcite{gear1971simultaneous}}}
\end{frame}

\subsubsection{Index Reduction}

\begin{frame}{Solution of \aclp{DAE}}{The Index \dots Or better Yet, The Indices}
  The index is a ``\textbf{measure of difficulty}'' in the solution of the \acsp{DAE}
  \begin{itemize}
    \item Different approaches in classifying such difficulties led to different \textbf{index concepts} \\
    \centering{\small\begin{tabular}{ccc}
      \hi{differential} & structural & tractability \\
      geometrical & perturbation & strangeness
    \end{tabular}}
    \item \raggedright Some indices \textbf{coincide} under proper regularity conditions
  \end{itemize}
  \vspace{0.25em}
  \uncover<2->{\begin{bbox}[The Differential Index]
    It is the minimum number of differentiations required to transform a \acs{DAE} system into its underlying \acs{ODE} system.
  \end{bbox}}
  \vspace{0.25em}
  \scriptsize{\fullcite{mehrmann2015index}}
\end{frame}

\begin{frame}{Solution of \aclp{DAE}}{Index Reduction}
  \begin{bbox}[Index Reduction in the Solution Process]
    \vspace{-1.0em}
    \begin{equation*}
      \begin{array}{c}
        \text{High-index} \\
        \text{\acs{DAE}}
      \end{array}
      \quad \autorightarrow{\text{index}}{\text{reduction}} \quad
      \begin{array}{c}
        \text{\acs{ODE}} \\
        \text{or} \\
        \text{Index-1 \acs{DAE}} \\
      \end{array} {\!+~\text{Invariants}}
      \quad \autorightarrow{\text{numerical}}{\text{integration}} \quad
      \begin{array}{c}
        \text{Solution of} \\
        \text{original \acs{DAE}}
      \end{array}
    \end{equation*}
  \end{bbox}
  \vspace{1.0em}
  Index reduction is performed
  \begin{itemize}
    %\item \textbf{prior} to numerical integration
    \item by expressions \textbf{manipulation} and \textbf{differentiation}
    \item through \textbf{symbolic computation} or \textbf{numerically} (with \emph{automatic differentiation})
  \end{itemize}
  \vspace{0.5em}
    \hic{No simple recipe exists!}
  \centering{\small\begin{tabular}{cc}
    Structural methods & Projector-based analysis
  \end{tabular}}
\end{frame}

\begin{frame}{Solution of \aclp{DAE}}{Differential and Algebraic Equations Separation}
  \uncover<1->{\hic{How do we reduce the index?}
  Exploit an idea common to many index concepts!
  \begin{enumerate}
    \item \textbf{Separate} the differential and algebraic equations
    \item \textbf{Differentiate} the algebraic equations
    \item \textbf{Repeat} \boxednumber{1}--\boxednumber{2} until necessary (index-1 \acs{DAE} or \acs{ODE} is reached)
  \end{enumerate}}
  \vspace{0.5em}
  \uncover<2->{\centering{Easy to say, not so easy to do \dots} \\
  \vspace{0.5em}
  \setlength{\tabcolsep}{3.5em}
  \small\begin{tabular}{cc}
    \hi{Numerically}                                             & \hi{Symbolically} \\
    \textcolor{mycolor5!95!black}{Numerical linear algebra}      & \textcolor{mycolor2!95!black}{Symbolic linear algebra} \\
    \textcolor{mycolor3!95!black}{Automatic differentiation}     & \textcolor{mycolor5!95!black}{Symbolic differentiation} \\
    \textcolor{mycolor2!95!black}{Track of the separators chain} & \textcolor{mycolor5!95!black}{Just manipulate the equations}
  \end{tabular}} \\[0.5em]
  \uncover<1->{\raggedright\scriptsize{\fullcite{mehrmann2015index}}}
\end{frame}

\subsection{A New Algorithm for Index Reduction}

\begin{frame}{Solution of \aclp{DAE}}{A New Algorithm for Index Reduction}
  \uncover<1->{\hic{How can we improve the state-of-the-art?}
  Exploit the \textbf{equation separation} concept in a \textbf{full-symbolic} framework!
  \begin{itemize}
    \item[\raisebox{-1pt}{\scalebox{0.8}{\faQuestion}}] \textbf{No} a priori knowledge of the system structure
    \item[\raisebox{-1pt}{\scalebox{0.8}{\faPaperPlane}}] \textbf{Basic} use of symbolic computation and linear algebra
  \end{itemize}
  \vspace{0.5em}
  \begin{columns}
    \centering
    \begin{column}[t]{0.25\textwidth}
      \centering
      \hi{Differentiation} \\[0.25em]
      $\dfrac{\text{d}}{\text{d}\textbf{x}}$ \vspace*{0.15cm}
    \end{column}
    \begin{column}[t]{0.25\textwidth}
      \centering
      \hi{Factorization} \\
      LU \\ Fraction-Free LU
    \end{column}
  \end{columns}}
  \vspace{0.5em}
  The \textbf{software tools} \\
  \begin{itemize}\small
    \item[\raisebox{0pt}{\scalebox{1.0}{\faCanadianMapleLeaf}}] \textbf{\Maple{}} for symbolic manipulation \\
    \item[\raisebox{0pt}{\scalebox{0.8}{\faAngleRight\hspace{-0.1em}\faAngleRight\hspace{-0.1em}\faAngleRight}}] \textbf{\Matlab{}} for numerical evaluation
  \end{itemize}
\end{frame}

% That's all Folks!

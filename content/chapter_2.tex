%!TEX root = main.tex

\section{\aclp{DAE}}

\begin{frame}{\aclp{DAE}}{A Brief Introduction}
  \begin{columns}
    \begin{column}{0.65\textwidth}
      \acp{DAE} are \dots
      \begin{itemize}
        \item a generalization of \acp{ODE} \dots
        \begin{equation*}
          \begin{array}{c}
            \text{\hi{\acp{ODE}}} \\
            \mxp = \m{f}(\mx, t)
            \phantom{~}
          \end{array}
          \qquad\qquad
          \begin{array}{c}
            \text{\hi{\acp{DAE}}} \\
            \mF = \m{0}
          \end{array}
        \end{equation*}
        \dots with $\m{JF}_{\mxp}$ possibly singular. \\[1.0em]
        \item state-of-the-art in the modeling of dynamical systems:
        \begin{itemize}
          \item differential equations describe the system's \textbf{dynamics};
          \item algebraic equations constrain the system to a \textbf{manifold}.
        \end{itemize}
      \end{itemize}
    \end{column}
    \begin{column}{0.35\textwidth}
      \centering
      \small
      \begin{tikzpicture}[scale=0.5]
        \fill (-1.5,0) rectangle (1.5,0.1);
        \draw[tx_sl_color!10] (0,0) -- (-90:4) node [fill, circle]{};
        \draw[tx_sl_color!10] (0,0) -- (-120:4) node [fill, circle]{};
        \draw[fg_sl_color, thick, dashed] (2.8284, -2.8284) arc (-45:-135:4);
        \draw (0,0) -- (-60:4) node[fill, circle](m){};
        \node at (m) [below, yshift=-2mm] {$m$};
        \node at (1.2142, -1.4142) {$\ell$};
      \end{tikzpicture}
      \begin{equation*}
        \begin{array}{c}
          \text{\ac{ODE}} \\ \text{model}
        \end{array}
        \begin{cases}
          \theta^\prime = \omega \\
          \omega^\prime = -\dfrac{g}{\ell} \sin(\theta)
        \end{cases}
      \end{equation*}
      \begin{equation*}
        \begin{array}{c}
          \text{\ac{DAE}} \\ \text{model}
        \end{array}
        \begin{cases}
          x^\prime = u \\
          y^\prime = v \\
          u^\prime = -2x\lambda \\
          v^\prime = -2y\lambda - gm \\
          \, \textcolor{fg_sl_color}{0 = x^2 + y^2 - \ell^2}
        \end{cases}
      \end{equation*}
    \end{column}
  \end{columns}
\end{frame}

\subsection{Solution of \aclp{DAE}}

\begin{frame}{Solution of \aclp{DAE}}{\underline{\acp{DAE} are not \acp{ODE} cit. Linda Petzold}}
  \vspace{-1.5cm}\hspace{4.25cm}\begin{tikzpicture}[scale=0.6]
    \draw[fg_sl_color, ultra thick, -stealth] (0.0,0.0) -- (-1.0,1.0);
  \end{tikzpicture} \\[1.0em]

  Is the solution of \acp{DAE} a straightforward extension of \acp{ODE}?
  \begin{itemize}
    \item \textbf{No}, otherwise we would not have been here.
    \item Numerical integration of \acp{DAE} is \textbf{challenging} due to the mixed differential and algebraic nature of the equations ($\m{JF}_{\mxp}$ is singular).
    \item An \textbf{analysis} and \textbf{reformulation} of the \ac{DAE} system is typically required prior to numerical integration.
  \end{itemize}
  \vspace{1.5em}
  \begin{columns}
    \begin{column}[t]{0.5\textwidth}
      \centering
      \hi{\large Direct Discretization}
    \end{column}
    \begin{column}[t]{0.5\textwidth}
      \centering
      \hi{\large Index Reduction}
    \end{column}
  \end{columns}
  \vspace{1.5em}
  \centering{\hi{\large \dots or a mix between the two!}}
\end{frame}

\subsubsection{Direct Discretization}

\begin{frame}{Solution of \aclp{DAE}}{Direct Discretization}
  \begin{bbox}[The Fundamental Idea]
    Given $\mF = \m{0}$, approximate $\mx$ and $\mxp$ by a \textbf{discretization formula}, \eg{}, a \emph{finite difference scheme} or a \emph{polynomial quadrature}.
  \end{bbox}
  \begin{itemize}
    \item This discretization can be written as an \textbf{implicit Runge-Kutta} method
    \begin{equation*}
      \begin{array}{l}
        \m{F}\left(\m{x}_{k} + h_k\displaystyle\sum_{j=1}^{s}a_{ij} \m{K}_j, \m{K}_i, t_{k} + c_i h_k\right) = \m{0} \, \text{,} \\[1em]
        \m{x}_{k+1} = \m{x}_{k} + h_k\displaystyle\sum_{i=1}^{s} b_i \m{K}_i \, \text{,}
      \end{array}
    \end{equation*}
    and can be applied to \acp{DAE} directly. If the matrix $\m{A} = (a_{ij})$ is non-singular (along with other conditions/assumption), the $\m{K}_i$ can be solved for at each step.
    \item A prominent example is the \textbf{Radau collocation method} family.
  \end{itemize}
\end{frame}

\subsubsection{Index Reduction}

\begin{frame}{Solution of \aclp{DAE}}{Index Reduction}
  \begin{bbox}[The Fundamental Idea]
    The \textbf{index reduction} consists in transforming a system of \acp{DAE} into
    \begin{itemize}
      \item an equivalent \ac{ODE} system with invariants;
      \item a lower-index \ac{DAE} system, typically of index 1.
    \end{itemize}
  \end{bbox}
  %
  \begin{itemize}
    \item Important \textbf{pre-processing} step prior to numerical integration.
    \begin{center}\begin{minipage}{7.5cm}\begin{bbox}{}
      \centering
      \acp{DAE} $\quad \autorightarrow{\text{index}}{\text{reduction}} \quad \begin{array}{c}
        \text{\acp{DAE}} + \text{Invariants} \\
        \text{or} \\
        \text{\acp{ODE}} + \text{Invariants}
      \end{array}$
    \end{bbox}\end{minipage} \\[1.5em] \end{center}
    %
    \item The index reduction is typically (but not always) carried out through \textbf{symbolic manipulation}.
  \end{itemize}
\end{frame}

\begin{frame}{Solution of \aclp{DAE}}{The Index \dots Or better Yet, The Indices}
  The index is a ``measure of difficulty'' in the analytical or numerical treatment of the \acp{DAE} \dots
  \begin{itemize}
    \item Index can be reduced through \textbf{different techniques}, \eg{}, Pantelides' algorithm, structural analysis, projector-based methods, etc.
    \item Different approaches in the classification of such difficulties lead to \textbf{different index concepts}.
    \item The most common index concept is the \textbf{differentiation index}.
  \end{itemize}
  \begin{bbox}[The Differentiation Index]
    The minimum number of differentiations required to transform a \ac{DAE} into its underlying \ac{ODE}. (by deriving part of the original \ac{DAE} equations),    is termed the differential index of the \ac{DAE}, with the underlying \acp{ODE} possessing an index of 0
  \end{bbox}
\end{frame}

\begin{frame}{Solution of \aclp{DAE}}{Index Reduction}
  \begin{itemize}
    \item Index Reduction is a well-studied problem, but hard to carry out in practice due to software limitations for symbolic manipulation.
    \\[1.0em]
    \item We propose an algorithm for index reduction based on:
    \\[0.5em]
    \begin{enumerate}
      \item \textbf{none} \emph{a priori} knowledge of the system structure;
      \\[0.5em]
      \item \textbf{basic} symbolic linear algebra techniques.
    \end{enumerate}
    \begin{center}\begin{minipage}[t]{0.30\textwidth}\begin{block}{}
      \centering
      \hi{Factorization} \\[0.5em]
      LU \\ Fraction-Free LU \vspace*{0.15cm}
    \end{block}\end{minipage}
    \hspace*{2cm}
    \begin{minipage}[t]{0.30\textwidth}\begin{block}{}
      \centering
      \hi{Differentiation} \\[0.75em]
      $\dfrac{\text{d}}{\text{d}\textbf{x}}$ \vspace*{0.15cm}
    \end{block}\end{minipage} \\[1.75em] \end{center}
    \item \Maple{} for symbolic manipulation, and \Matlab{} for numerical integration of the reduced system.
  \end{itemize}
\end{frame}

% That's all Folks!
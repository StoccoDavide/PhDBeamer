%!TEX root = main.tex

\section{Symbolic Computation Essentials}

\begin{frame}
  \tableofcontents[currentsection]
\end{frame}

\begin{frame}{Symbolic Computation Essentials}{Introduction}
  \begin{itemize}
    \item Symbolic computation (or computer algebra) involves manipulating mathematical expressions in their \textbf{exact} form.
    \item Such an exact representation implies that, even when the size of the output is small, the intermediate expressions generated during a computation may grow unpredictably due to \textbf{expression swell}.
  \end{itemize}
  \begin{center}\begin{minipage}{\textwidth}\begin{bbox}[The Ideal Workflow]
    \centering%
    \begin{minipage}[c]{0.15\textwidth} \centering Input expression \end{minipage}%
    \begin{minipage}[c]{0.19\textwidth} \centering $\autorightarrow{\text{symbolic}}{\text{manipulation}}$ \end{minipage}%
    \begin{minipage}[c]{0.21\textwidth} \centering Output expression \\ + \\ Expression swell \end{minipage}%
    \begin{minipage}[c]{0.19\textwidth} \centering $\autorightarrow{\text{symbolic}}{\text{simplification}}$ \end{minipage}%
    \begin{minipage}[c]{0.21\textwidth} \centering Output expression \\ + \\ \xout{Expression swell} \end{minipage}%
  \end{bbox}\end{minipage}\end{center}
  \hic{All \acp{CAS} are sensitive to expression swell, some more than others!}
\end{frame}

\subsection{Expression Swell}

\begin{frame}{Expression Swell}{A Rough Definition}
  \begin{bbox}[Not-So-Formal Definition]
    Exponential growth of expression size during symbolic manipulation, which undermines \ac{CAS} efficiency and leads to high memory usage and slow computation.
  \end{bbox}
  \vspace{0.75em}
  There are two types of expression swell \dots
  \vspace{0.75em}
  \begin{columns}
    \begin{column}[t]{0.5\textwidth}
      \hi{\large Inherent \textcolor{black}{expression swell}} \\
      \begin{itemize}\small
        \item The size of expressions can not be lowered due to no simplification rules being applicable.
        \item Can not always be avoided and requires changes in the problem formulation.
      \end{itemize}
    \end{column}
    \begin{column}[t]{0.5\textwidth}
      \hi{\large Intermediate \textcolor{black}{expression swell}} \\
      \begin{itemize}\small
        \item During the middle stages of a calculation the size of intermediate expressions can expand substantially \emph{en route} to a comparatively simple final result.
        \item Strongly related to the \ac{CAS}'s simplification capabilities.
      \end{itemize}
    \end{column}
  \end{columns}
\end{frame}

\subsection{\acl{LEM}}

\begin{frame}{\acl{LEM}}{Hierarchical Representation}
  The \textbf{hierarchical representation} of expressions is carried out through \textbf{veiling variables}, which are defined as \dots
  \begin{equation*}
    \m{f}(\mx) \quad \autorightarrow{\text{veiling}}{\text{variables}} \quad \m{f}(\mx, \m{v}) \qquad \text{where} \qquad \m{v}(\mx) = \begin{bmatrix}
      v_{1}(\mx) \\
      v_{2}(v_{1}, \mx) \\
      \vdots \\
      v_{n}(v_{1}, \dots, v_{n-1}, \mx)
    \end{bmatrix}
  \end{equation*}
  The process is composed of two actions \dots
  \begin{enumerate}
    \item a large expression can be \textbf{veiled} (stored) in a \textbf{veiling variable};
    \item veiling variables can be \textbf{unveiled} (substitute back) the veiling variables into the expression to recover the original form.
  \end{enumerate}
  \vspace{0.5em}
  \hic{The complexity remains but the \ac{CAS} can not see it!}
\end{frame}

\begin{frame}{\acl{LEM}}{An Example of \acs{LEM} through Hierarchical Representation}
  Let us consider an expression of the form
  \begin{equation*}
    \textcolor{mycolor1}{\underline{2xy(1+y)}} \cdot \text{f}(x,y) +
    \textcolor{mycolor2}{\underline{5z(3a+z)}} \cdot \text{g}(y) +
    \textcolor{mycolor3}{\underline{3c(xy+z)}} \cdot \text{h}(x,z) \text{.}
  \end{equation*}
  One of its possible hierarchical representations is
  \begin{equation*}
    \textcolor{mycolor1}{v_1} \cdot \text{f}(x,y) +
    \textcolor{mycolor2}{v_2} \cdot \text{g}(y) +
    \textcolor{mycolor3}{v_3} \cdot \text{h}(x,z) \text{,}
  \end{equation*}
  where the veiling variables are
  \begin{equation*}
    \textcolor{mycolor1}{v_1} = 2x(1+y) \text{,}
    \qquad
    \textcolor{mycolor2}{v_2} = 5z(3a+z) \text{,}
    \qquad \text{and} \qquad
    \textcolor{mycolor3}{v_3} = 3c(xy+z) \text{.}
  \end{equation*}
\end{frame}

\begin{frame}{\acl{LEM}}{Measuring Expression Complexity}
  Easy to implement, but \dots
  \vspace{1.5em}
  \begin{columns}
    \begin{column}[t]{0.5\textwidth}
      \hi{\large How to measure symbolic expression complexity?}
      \begin{itemize}
        \normalsize
        \item \textbf{Length in characters}: already used by other authors, not reliable;
        \item \textbf{Directed acyclic graph}: measures the number of nodes and edges.
        \item \textbf{Computational cost}: insensible to expression internal representation. \\[0.25em]
        \emph{We can roughly predict the swelling related to elementary mathematical operations!}
      \end{itemize}
    \end{column}
    \begin{column}[t]{0.5\textwidth}
      \hi{\large How to choose the right level of expression complexity?}
      \begin{itemize}
        \normalsize
        \item \textbf{Prediction}: based on the specific operations to be performed;
        \item \textbf{Trial and error}: no general rule, depends on the software.
      \end{itemize}
    \end{column}
  \end{columns}
\end{frame}

\begin{frame}{\acl{LEM}}{The \ac{LEM} Package}
  Easy to implement, but \dots
  \begin{enumerate}
    \item How to measure symbolic expression complexity?
    \begin{itemize}
      \item \textbf{Length in characters}: already used by other authors, not reliable;
      \item \textbf{Computational cost}: insensible to expression internal representation;
      \item \textbf{Directed acyclic graph}: measures the number of nodes and edges, but not the complexity.
    \end{itemize}
    \item How to choose the right level of expression complexity?
    \begin{itemize}
      \item \textbf{Trial and error}: no general rule (depends on the software).
    \end{itemize}
  \end{enumerate}
  %
  \begin{bbox}[Large Expression Management \Maple{} package]
    The \ac{LEM} object-oriented implementation is freely available on GitHub! \\
    \centering \url{https://github.com/StoccoDavide/LEM}
  \end{bbox}
\end{frame}

\subsection{Symbolic Matrix Factorization}

\begin{frame}{Symbolic Matrix Factorization}
  \textbf{Numeric} factorization is a fundamental operation in numerical linear algebra. It is used to \dots
  \begin{itemize}
    \item solve linear systems of equations without inverting the original matrix;
    \item reduce the number of computations and increase the stability;
    \item provide insights into the properties of the original matrix (\ie{}, rank and determinant).
  \end{itemize}
  \vspace{1.0em}
  \hic{Does this also apply to the symbolic computation? Yes!}
  \vspace{1.0em}
  However\dots
  \begin{enumerate}
    \item the output is not guaranteed to be stable once numerically evaluated;
    \begin{itemize}
      \item[] \textbf{How to ensure the stability of the symbolic factorization?}
    \end{itemize}
    \item symbolic expressions tend to grow during manipulation.
    \begin{itemize}
      \item[] \textbf{How to manage large symbolic expressions?}
    \end{itemize}
  \end{enumerate}
\end{frame}

\begin{frame}{Symbolic Matrix Factorization}
  \begin{itemize}
    \item Symbolic matrix factorization is used for the computation of the \textbf{kernel}, \ie{}, for separating the system differential and algebraic parts.
    \item There are a multitude of matrix factorizations, we choose \textbf{\ac{LU}} and \textbf{\ac{FFLU}} because of \dots
    \begin{enumerate}
      \item capability of preserving \textbf{sparsity} with \textbf{minimum degree} approach;
      \item limited expression swell with the \textbf{full-pivoting} strategy;
      \item improved numerical stability with custom symbolic \textbf{pivoting strategy};
      \item depending on the formulation, improved numerical stability with \textbf{fraction-free} factorization.
    \end{enumerate}
    \item \Maple{} matrix factorizations are \textbf{sensitive to expression swell}.
    \item We developed a symbolic matrix factorization toolbox capable of dealing with expression swell.
    \begin{enumerate}
      \item At specific steps of the algorithm \textbf{veiling variables} are introduced not to increase the size of the expressions.
      \item \ac{LU}, \ac{FFLU}, \ac{QR}, and Gauss-Jordan factorizations are available.
    \end{enumerate}
\end{itemize}
  %
  \begin{bbox}[Linear Algebra Symbolic Toolbox \Maple{} package]
  The LAST object-oriented implementation is freely available on GitHub! \\
  \centering \url{https://github.com/StoccoDavide/LAST}
\end{bbox}
\end{frame}

\begin{frame}{Symbolic Matrix Factorization}{\acf{LU}}
  The ``standard'' in numerical linear algebra \dots
  \begin{bbox}[Full-Pivoting \acs{LU} Factorization]
    Given a matrix $\m{A} \in \mathbb{R}^{m \times n}$, with $m \geq n$, the full-pivoting \ac{LU} decomposition is defined as the process of decomposing $\m{A}$ into the product of
    %
    \begin{itemize}
      \item a $\m{L} \in \mathbb{R}^{m \times m}$ lower-triangular matrix with all diagonal entries equal to $1$;
      \item a $\m{U} \in \mathbb{R}^{m \times n}$ upper-triangular matrix;
      \item a $\m{P} \in \mathbb{R}^{m \times m}$ and a $\m{Q} \in \mathbb{R}^{n \times n}$ matrices for rows and columns permutation;
    \end{itemize}
    %
    such that $\m{P}\m{A}\m{Q} = \m{L}\m{U}$.
  \end{bbox}
  \dots It carries the minimum number of operations!
\end{frame}

\begin{frame}{Symbolic Matrix Factorization}{\acf{FFLU}}
  Trying to guarantee exact divisions on factors \dots
  \begin{bbox}[Full-Pivoting \acs{FFLU} Factorization]
    Given a matrix $\m{A} \in \mathbb{R}^{m \times n}$, with $m \geq n$, the full-pivoting \ac{FFLU} decomposition is defined as the process of decomposing $\m{A}$ into the product of
    %
    \begin{itemize}
      \item a lower-triangular matrix $\m{L} \in \mathbb{R}^{m \times m}$ with all diagonal entries equal to $1$;
      \item \textcolor{fg_sl_color}{a diagonal matrix $\m{D} \in \mathbb{R}^{m \times n}$;}
      \item an upper-triangular matrix $\m{U} \in \mathbb{R}^{m \times n}$;
      \item a $\m{P} \in \mathbb{R}^{m \times m}$ and a $\m{Q} \in \mathbb{R}^{n \times n}$ matrices for rows and columns permutation;
    \end{itemize}
    %
    such that $\m{P}\textcolor{fg_sl_color}{\m{D}}\m{A}\m{Q} = \m{L}\m{U}$.
  \end{bbox}
  \dots same effectiveness of \ac{LU} decomposition, but with a more complex implementation!
\end{frame}

% That's all Folks!
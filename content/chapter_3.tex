%!TEX root = main.tex

\section{Symbolic Computation Essentials}

\begin{frame}{Symbolic Computation Essentials}{Introduction}
  In symbolic computation \dots
  \begin{itemize}
    \item mathematical expressions are manipulated in their \textbf{exact} form.
    \item the \ac{CAS} is the tool that performs the symbolic computation.
    \item during computation expressions may \textbf{grow} unpredictably! % even if the size of the output is small,
  \end{itemize}
  \begin{center}\begin{minipage}{\textwidth}\begin{bbox}[Inside a \acs{CAS}]
    \centering%
    \begin{minipage}[c]{0.15\textwidth} \centering Input expression \end{minipage}%
    \begin{minipage}[c]{0.19\textwidth} \centering $\autorightarrow{\text{symbolic}}{\text{manipulation}}$ \end{minipage}%
    \begin{minipage}[c]{0.21\textwidth} \centering Output expression \\ + \\ Expression swell \end{minipage}%
    \begin{minipage}[c]{0.19\textwidth} \centering $\autorightarrow{\text{symbolic}}{\text{simplification}}$ \end{minipage}%
    \begin{minipage}[c]{0.21\textwidth} \centering Output expression \\ + \\ \textcolor{fg_sl_color!25}{\sout{Expression swell} }\end{minipage}%
  \end{bbox}\end{minipage}\end{center}
  \hic{All \acs{CAS} are sensitive to expression swell!}
\end{frame}

\subsection{Expression Swell}

\begin{frame}{Expression Swell}{A Rough Definition}
  \begin{bbox}[Expression Swell]
    Exponential growth of expression size during symbolic manipulation, which undermines \ac{CAS} efficiency and leads to high memory usage and slow computation.
  \end{bbox}
  \vspace{0.5em}
  \centering{\begin{tikzpicture}[scale=0.5]
    \node at (0,0) {There are two types of expression swell \dots};
    \draw[fg_sl_color, thick, -stealth] (+2,-0.75) -- (+3,-1.75);
    \draw[fg_sl_color, thick, -stealth] (-2,-0.75) -- (-3,-1.75);
  \end{tikzpicture}} \\
  \begin{columns}
    \begin{column}[t]{0.5\textwidth}
      \raggedleft{\hi{\large Inherent}~~~~~} \\
      \begin{itemize}\small
        \item No applicable simplification \textbf{rule}.
        \item Require changes in the problem \textbf{formulation}.
      \end{itemize}
    \end{column}
    \begin{column}[t]{0.5\textwidth}
      \hi{\large Intermediate} \\
      \begin{itemize}\small
        \item Depends on \ac{CAS}'s simplification \textbf{capabilities}.
        \item During the middle stages of a calculation the size of intermediate expressions can expand substantially \emph{en route} to a comparatively simple final result.
      \end{itemize}
    \end{column}
  \end{columns}
\end{frame}

\subsection{\acl{LEM}}

\begin{frame}{\acl{LEM}}{Hierarchical Representation}
  Let us consider an expression of the form
  \begin{equation*}
    \textcolor{mycolor1}{\underline{2xy(1+y)}} \cdot f(x,y) +
    \textcolor{mycolor2}{\underline{5z(3a+z)}} \cdot g(y) +
    \textcolor{mycolor3}{\underline{3c(xy+z)}} \cdot h(x,z) \text{.}
  \end{equation*}
  One of its possible hierarchical representations is
  \begin{equation*}
    \textcolor{mycolor1}{v_1} \cdot f(x,y) +
    \textcolor{mycolor2}{v_2} \cdot g(y) +
    \textcolor{mycolor3}{v_3} \cdot h(x,z) \text{,}
  \end{equation*}
  where the veiling variables are
  \begin{equation*}
    \textcolor{mycolor1}{v_1} = 2x(1+y) \text{,}
    \qquad
    \textcolor{mycolor2}{v_2} = 5z(3a+z) \text{,}
    \qquad \text{and} \qquad
    \textcolor{mycolor3}{v_3} = 3c(xy+z) \text{.}
  \end{equation*}
\end{frame}

\begin{frame}{\acl{LEM}}{Hierarchical Representation}
  The \textbf{hierarchical representation} of expressions is carried out through \textbf{veiling variables}, which are defined as \dots
  \begin{equation*}
    \m{f}(\mx) \quad \autorightarrow{\text{veiling}}{\text{variables}} \quad \m{f}(\mx, \m{v})
    \qquad \text{where} \qquad
    \m{v}(\mx) = \begin{bmatrix}
      v_{1}(\mx) \\
      v_{2}(v_{1}, \mx) \\
      \vdots \\
      v_{n}(v_{1}, \dots, v_{n-1}, \mx)
    \end{bmatrix}
  \end{equation*}
  The process is composed of two actions \dots
  \begin{enumerate}
    \item a large expression can be \textbf{veiled} (stored) in a \textbf{veiling variable};
    \item veiling variables can be \textbf{unveiled} (substitute back) the veiling variables into the expression to recover the original form.
  \end{enumerate}
  \vspace{0.5em}
  \hic{The complexity remains but the \ac{CAS} can not see it!}
\end{frame}

\begin{frame}{\acl{LEM}}{Measuring Expression Complexity}
  Easy to implement, but \dots
  \vspace{1.5em}
  \begin{columns}
    \begin{column}[t]{0.5\textwidth}
      \hi{\large How to measure symbolic expression complexity?}
      \begin{itemize}
        \normalsize
        \item \textbf{Length in characters}: same expression can have different lengths.
        \emph{Not reliable!}
        \item \textbf{Directed acyclic graph}: measures the number of nodes and edges.
        \emph{Not easily accessible!}
        \item \textbf{Computational cost}: insensible to expression internal representation. \\[0.25em]
        \emph{Alredy implemented in \Maple{}!}
      \end{itemize}
    \end{column}
    \begin{column}[t]{0.5\textwidth}
      \hi{\large How to choose the right level of expression complexity?}
      \begin{itemize}
        \normalsize
        \item \textbf{Prediction}: based on the specific operations to be performed;
        \item \textbf{Trial and error}: no general rule, depends on the software.
      \end{itemize}
    \end{column}
  \end{columns}
\end{frame}

\subsection{Symbolic Matrix Factorization}

\begin{frame}{Symbolic Matrix Factorization}
  \textbf{Numeric} factorization is a fundamental operation in numerical linear algebra. It is used to \dots
  \begin{itemize}
    \item solve linear systems of equations (inverting the original matrix is not recommended);
    \item provide insights into the properties of the original matrix (\ie{}, rank and determinant).
  \end{itemize}
  \vspace{1.0em}
  \hic{Does this also apply to the symbolic computation? \textcolor{black}{Yes!}}
  \vspace{1.0em}
  However\dots
  \begin{enumerate}
    \item the output is not guaranteed to be stable once numerically evaluated;
    \begin{itemize}
      \item[] \textbf{How to ensure the stability of the symbolic factorization?}
    \end{itemize}
    \item symbolic expressions tend to grow during manipulation.
    \begin{itemize}
      \item[] \textbf{How to manage large symbolic expressions?}
    \end{itemize}
  \end{enumerate}
\end{frame}

\begin{frame}{Symbolic Matrix Factorization}
  \begin{itemize}
    \item There are many factorizations, we choose \dots
    \begin{itemize}
      \item \textbf{\ac{LU}};
      \item \textbf{\ac{FFLU}};
    \end{itemize}
    because of \dots
    \begin{enumerate}
      \item capability of preserving \textbf{sparsity} with \textbf{minimum degree} approach;
      \item limited expression swell with the \textbf{full-pivoting} strategy;
      \item improved numerical stability with custom symbolic \textbf{pivoting strategy};
      \item depending on the formulation, improved numerical stability with \textbf{fraction-free} factorization.
    \end{enumerate}
    \item \Maple{} matrix factorizations are \textbf{sensitive to expression swell}.
    \item We developed \ac{LAST} \dots
    \begin{enumerate}
      \item At specific steps of the algorithm \textbf{veiling variables} are introduced not to increase the size of the expressions.
    \end{enumerate}
  \end{itemize}
\end{frame}

\begin{frame}{Symbolic Matrix Factorization}{Symbolic Pivoting Strategy}
  \vspace{-1.75em}
  \hic{How to ensure numerical stability? \textcolor{black}{With a pivoting strategy!}}
  \vspace{-0.75em}
  \begin{columns}
    \begin{column}[c]{0.5\textwidth}
      A good symbolic pivoting strategy should \dots
      \begin{itemize}
        \item if numeric select the \textbf{biggest} numerical pivot:
        \begin{itemize}
          \item[] Improve numerical stability;
        \end{itemize}
        \item if not, select the \textbf{simplest} pivot element:
        \begin{itemize}
          \item[] Limit expression swell;
        \end{itemize}
        \item explore \textbf{all entries} for the best choice:
        \begin{itemize}
          \item[] Full-pivoting strategy.
        \end{itemize}
      \end{itemize}
    \end{column}
    \begin{column}[c]{0.6\textwidth}
      \begin{algorithmic}\scriptsize
        %\State \textbf{Require:} A $m \times n$ matrix $\m{A}$.
        %\State \phantom{\textbf{Require:}} The $k$-th pivoting stage.
        \Procedure{SymbolicPivoting}{$\m{A}$, $k$} %\Comment{Symbolic pivoting procedure for the $k$-th pivot}
          \State $\m{d}^r, \, \m{d}^c \gets \text{ComputeDegrees}(\m{A})$ %\Comment{Calculate the row and column degrees of $\m{A}$}
          \For{$i$ \textbf{from} $k$ \textbf{to} $m$} %\Comment{Iterate over the rows}
            \For{$j$ \textbf{from} $k$ \textbf{to} $n$} %\Comment{Iterate over the columns}
              \State $D_{ij} \gets \infty$ %\Comment{Set the combined degree matrix to infinity}
              \IfThen{$A_{ij} \neq 0$}
              {\hlc{$D_{ij} \gets d^r_{i} \, \max(0, \, d^c_j-1) + d^c_j \, \max(0, \, d^r_i-1)$}} %\Comment{The degree}
            \EndFor
          \EndFor
          \State \hlc{$\mathcal{P} \gets \text{Sort}(\m{D})$} %\Comment{Find the permutation that sorts the pivots list by degree cost}
          \State $q, \, l \gets \, 0, \, 0$  ~ and ~ %\Comment{Initialize the temporary pivot row and column indices}
          $p, \, p_c, \, p_n \gets \infty, \, \infty, \, \infty$ %\Comment{Initialize the pivot value, complexity and numerical value}
          \For{\textbf{all} $(i, j)$ \textbf{in} $\mathcal{P}$} %\Comment{Iterate on the permutation set}
            \IfThen{$p_c \neq \infty$ \textbf{and} $D_{ij} > D_{ql}$}{\textbf{break}} %\Comment{No more good pivots to check}
            \State $t \gets A_{ij}$ %\Comment{Get the pivot value}
            \IfThen{$\text{Signature}(t) = 0$}{\textbf{continue}} %\Comment{Skip the next pivot}
            \State $t \gets \text{Simplify}(t)$ ~ and ~ %\Comment{Try to simplify the pivot expression}
            \hlc{$t_c \gets \text{ExpressionComplexity}(t)$} ~ and ~ %\Comment{Calculate the computational cost of the pivot}
            $t_n \gets \infty$ %\Comment{Set the default numerical value of the pivot to infinity}
            \IfThen{$t$ is numeric}{\hlc{$t_n \gets \max(1, \, \text{abs}(t))$}} %\Comment{Set the numerical value of the pivot}
            \If{\hlc{$t_c < p_c$} \textbf{or} (\hlc{$t_c = p_c$} \textbf{and} \hlc{$t_n > p_n$})} %\Comment{If the pivot is better than the current one}
              \State $q, \, l \gets i, \, j$ ~ and ~ %\Comment{Update the best pivot row and column indices}
              $p, \, p_c, \, p_n \gets t, \, t_c, \, t_n$ %\Comment{Save the best pivot value, complexity and numerical value}
            \EndIf
          \EndFor
          \State \textbf{return} $p, \, q, \, l$ %\Comment{The $k$-th pivot and its position}
        \EndProcedure
      \end{algorithmic}
    \end{column}
  \end{columns}
\end{frame}

\begin{frame}{Symbolic Matrix Factorization}{\ac{LEM} During Factorization}
  \vspace{-2.5em}
  \hic{How to manage large symbolic expressions? \textcolor{black}{By using \acs{LEM}!}}
  \begin{columns}
    \hspace{1.0em}
    \begin{column}[c]{0.425\textwidth}
      Veils are introduced during \dots
      \begin{itemize}
        \item \textbf{Factorization step}: \\
        \begin{itemize}\small
          \item[] Gaussian elimination;
          \item[] Schur complement computation.
        \end{itemize}
        \item \textbf{Solution step}: \\
        \begin{itemize}\small
          \item[] Forward substitution;
          \item[] Backward substitution.
        \end{itemize}
      \end{itemize}
    \end{column}
    \begin{column}[c]{0.4\textwidth}
      \vspace{-1.0em}
      \hspace{-1.0em}
      \begin{algorithmic}\scriptsize
        %\State \textbf{Require:} A $m \times n$ matrix $\m{A}$.
        \Procedure{LU}{$\m{A}, \, k$} %\Comment{Symbolic full-pivoting \ac{LU} procedure}
          \State $\m{M} \gets \m{A}$ %\Comment{Initialize the matrix $\m{M}$}
          \State $rnk \gets \min(m, \, n)$ %\Comment{Initialize the rank of $\m{M}$}
          \For{$k$ \textbf{from} $1$ \textbf{to} $rnk$} %\Comment{Perform Gaussian elimination}
            \State $p, \, q, \, l \gets \text{SymbolicPivoting}(\m{M}, \, k)$ %\Comment{Find the best pivot for the $k$-th step}
            \If{$p = 0$} %\Comment{Check for null pivot}
              \State $rnk \gets k-1$ ~ and ~ %\Comment{The rank of $\m{M}$ is $k-1$}
              \textbf{break} %\Comment{The matrix is singular}
            \EndIf
            \State $\m{r}_k, \, \m{c}_k \gets \, q, \, l$ %\Comment{Store the pivot row and column indices}
            \State $\m{M} \gets \text{SwapRowsCols}(\m{M}, \, k, \, q,  \, l)$ %\Comment{Swap the $k$-th and $q$-th rows}
            \For{$i$ \textbf{from} $k+1$ \textbf{to} $m$} %\Comment{Compute the $k$-th column of $\m{L}$}
              \State \hlc{$M_{kk} \gets \text{Veil}(M_{kk})$} %\Comment{Veil the $k$-th pivot}
              \State \hlc{$M_{ik} \gets \text{Veil}(\text{Normalizer}(M_{ik}/M_{kk}))$} %\Comment{Normalize the $k$-th pivot}
              \For{$j$ \textbf{from} $k+1$ \textbf{to} $n$} %\Comment{Compute the $k$-th row of $\m{U}$}
                \State \hlc{$M_{ij} \gets \text{Veil}(\text{Normalizer}(M_{ij} - M_{ik}M_{kj}))$} %\Comment{Finalize the Schur complement}
              \EndFor
            \EndFor
          \EndFor
          \State $\m{P}, \, \m{Q} \gets \text{PermutationMatrices}(\m{r}, \, \m{c})$ %\Comment{Compute the permutation matrices}
          \State $\m{L}, \, \m{U} \gets \text{LowerUpperMatrices}(\m{M})$ %\Comment{Extract the lower-triangular part of $\m{M}$}
          \State \textbf{return} $\m{L}, \, \m{U}, \, \m{P}, \, \m{Q}, \, \m{r}, \, \m{c}, \, rnk$ %\Comment{Return the factors and the rank of $\m{A}$}
        \EndProcedure
      \end{algorithmic}
    \end{column}
    \begin{column}[c]{0.3\textwidth}
      \vspace{-1.0em}
      \begin{algorithmic}\scriptsize
        %\State \textbf{Require:} The \ac{LU} factors $\m{L}$, $\m{U}$, $\m{P}$, $\m{Q}$, and a vector $\m{b}$.
        \Procedure{SolveLU}{$\m{L}, \, \m{U}, \, \m{P}, \, \m{Q}, \, \m{b}$} %\Comment{Solve the linear system $\m{A}\m{x} = \m{b}$}
          \State $\m{y} \gets \m{P}\m{b}$ %\Comment{Apply the permutation matrix $\m{P}$ to the vector $\m{b}$}
          \State $m, \, n \gets \text{Size}(\m{L})$ %\Comment{Get the size of $\m{L}$}
          \For{$i$ \textbf{from} $2$ \textbf{to} $m$} %\Comment{Solve $\m{L}\m{y} = \m{P}\m{b}$}
            \State \hlc{$y_i \gets \text{Veil}(y_i - \sum\nolimits_{j=1}^{i-1} L_{ij}y_j)$} %\Comment{Perform forward substitution}
          \EndFor
          \State $x_n \gets \text{Veil}(y_n/U_{nn})$ %\Comment{Perform the first backward substitution}
          \For{$i$ \textbf{from} $n-1$ \textbf{to} $1$} %\Comment{Solve $\m{U}\m{x} = \m{y}$}
            \State \hlc{$x_i \gets \text{Veil}(y_i - {\sum\nolimits_{j=i+1}^{n}} U_{ij}x_j)$} %\Comment{Perform backward substitution}
            \State \hlc{$x_i \gets \text{Veil}(x_i/U_{ii})$}
          \EndFor
          \State $\m{x} \gets \m{Q}^\top\m{x}$ %\Comment{Apply the permutation matrix $\m{Q}^\top$ to the solution $\m{x}$}
        \EndProcedure
      \end{algorithmic}
    \end{column}
  \end{columns}
\end{frame}

% That's all Folks!

%!TEX root = main.tex

\section{Symbolic Computation Essentials}

\subsection{Expression Swell}

\begin{frame}{Symbolic Computation Essentials}{Expression Swell}
  \uncover<1->{In symbolic computation, expressions may \textbf{grow} unpredictably!}
  \uncover<2->{\begin{center}\begin{minipage}{\textwidth}\begin{bbox}[Inside a \acs{CAS}]
    \centering%
    \begin{minipage}[c]{0.15\textwidth} \centering Input expression \end{minipage}%
    \begin{minipage}[c]{0.19\textwidth} \centering $\autorightarrow{\text{symbolic}}{\text{manipulation}}$ \end{minipage}%
    \begin{minipage}[c]{0.21\textwidth} \centering Output expression \\ + \\ Expression swell \end{minipage}%
    \begin{minipage}[c]{0.19\textwidth} \centering $\autorightarrow{\text{symbolic}}{\text{simplification}}$ \end{minipage}%
    \begin{minipage}[c]{0.23\textwidth} \centering Output expression \\ + \\ \textcolor{mycolor2!95!black}{\raisebox{7.5pt}{\rotatebox{180}{\scriptsize\faQuestion}}Expression swell{\scriptsize\faQuestion}}\end{minipage}%
  \end{bbox}\end{minipage}\end{center}}
  \uncover<3->{\hic{All \acs{CAS} are sensitive to expression swell!}
  \centering{\small\begin{tabular}{cc}
    \textcolor{fg_sl_color}{\faCoins} High memory usage & Slow computation \textcolor{fg_sl_color}{\faHourglassHalf}
  \end{tabular}}} \\[0.5em]
  \centering{\begin{tikzpicture}[scale=0.5]
    \uncover<4->{\node at (0,0) {\textbf{Two} types of expression swell \dots};}
    \uncover<5->{\draw[fg_sl_color, thick, -stealth] (-2,-0.5) -- (-4.0,-1.5);
    \node at (-7,-2.5) {
      \begin{tabular}{c}
        \hi{\large ~~~~~~Inherent} \\
        \small{No applicable simplification \textbf{rule}} \\
        \small{Require changes in the problem \textbf{formulation}}
      \end{tabular}
    };}
    \uncover<6->{\draw[fg_sl_color, thick, -stealth] (+2,-0.5) -- (+4.0,-1.5);
    \node at (+7,-2.5) {
      \begin{tabular}{c}
        \hi{\large Intermediate} \\
        \small{Depends on \acs{CAS}'s simplification \textbf{capabilities}} \\
        \small{Intermediate growth, \textbf{en route} to a simple result}
      \end{tabular}
    };}
  \end{tikzpicture}}
\end{frame}

\subsection{\acl{LEM}}

\begin{frame}{\acl{LEM}}{Hierarchical Representation}
  Where progressively \textbf{hide} complexity in veiling variables
  \begin{equation*}
    \textcolor{mycolor1}{\underline{2xy(1+y)}} \cdot f(x,y) + \textcolor{mycolor2}{\underline{5z(3a+z)}} \cdot g(y) + \textcolor{mycolor3}{\underline{3c(xy+z)}} \cdot h(x,z)
  \end{equation*}
  \begin{equation*}
    \textcolor{mycolor5}{\underline{v_1 \cdot f(x,y)}} + \textcolor{mycolor4}{\underline{v_2 \cdot g(y) + v_3 \cdot h(x,z)}}
  \end{equation*}
  \begin{equation*}
    v_4 + v_5
  \end{equation*}
  where the \textbf{veiling variables} are
  \begin{align*}
    \textcolor{mycolor1}{v_1} &= 2x(1+y) \\
    \textcolor{mycolor2}{v_2} &= 5z(3a+z) \\
    \textcolor{mycolor3}{v_3} &= 3c(xy+z) \\
    \textcolor{mycolor4}{v_4} &= \textcolor{mycolor2}{v_2} \cdot g(y) + \textcolor{mycolor3}{v_3} \cdot h(x,z) \\
    \textcolor{mycolor5}{v_5} &= \textcolor{mycolor1}{v_1} \cdot f(x,y)
  \end{align*}
\end{frame}

\begin{frame}{\acl{LEM}}{Hierarchical Representation}
  \uncover<1->{\textbf{Hierarchical representation} of expressions is performed through \textbf{veiling variables}
  \begin{equation*}
    \m{f}(\mx) \quad \autorightarrow{\text{veiling}}{\text{variables}} \quad \m{f}(\mx, \m{v})
    \qquad \text{where} \qquad
    \m{v}(\mx) = \msmall{\begin{bmatrix}
      v_{1}(\mx) \\
      v_{2}(v_{1}, \mx) \\
      \vdots \\
      v_{n}(v_{1}, \dots, v_{n-1}, \mx)
    \end{bmatrix}}
  \end{equation*}}
  \uncover<2->{The process is made of two actions \dots
  \begin{enumerate}
    \item a large expression can be \textbf{veiled} in a veiling variable
    \item veils can be \textbf{unveiled} to recover the original form
  \end{enumerate}}
  \vspace{0.5em}
  \uncover<3->{\hic{Complexity remains but the \acs{CAS} can not see it! \textcolor{fg_sl_color}{\large \faGlasses}}}
\end{frame}

\begin{frame}{\acl{LEM}}{Measuring Expression Complexity}
  \centering{\begin{tikzpicture}[scale=0.5]
    \node at (0,0) {Easy to implement, but \dots};
    \uncover<1->{
      \draw[fg_sl_color, thick, -stealth] (-2,-1) -- (-4.0,-3);
      \draw (-0.5\textwidth,-3.25) node[below, text width=0.5\textwidth] {
      \hi{\large How to measure symbolic expression complexity?}
      \begin{itemize}
        \item \textbf{Length in characters} \\
        \begin{small}
          \quad \textcolor{mycolor2!95!black}{Same expression can have different lengths}
        \end{small}
        \item \textbf{Directed acyclic graph} \\
        \begin{small}
          \quad \textcolor{mycolor3!95!black}{Measures the number of nodes and edges}
        \end{small}
        \item \textbf{Computational cost} \\
        \begin{small}
          \quad \textcolor{mycolor5!95!black}{Insensible to internal representation}
        \end{small}
      \end{itemize}
    };}
    \uncover<2->{
      \draw[fg_sl_color, thick, -stealth] (+2,-1) -- (+4.0,-3);
      \draw (0.5\textwidth,-3.25) node[below, text width=0.5\textwidth] {
      \hi{\large What is the right level of expression complexity?}
      \begin{itemize}
        \item \textbf{Trial and error} \\
        \begin{small}
          \quad No general rule, depends on the software
        \end{small}
      \end{itemize}};
    }
  \end{tikzpicture}}
\end{frame}

\subsection{Symbolic Matrix Factorization}

\begin{frame}{Symbolic Matrix Factorization}{Considerations and Challenges}
  \hic{\faCanadianMapleLeaf~\Maple{} factorizations are sensitive to expression swell}
  \begin{itemize}
    \item<2-> \textbf{\acl{LU}~(\acs{LU})} and \textbf{\acl{FFLU}~(\acs{FFLU})} factorizations \dots \\
    \begin{small}
      \qquad preserve \textbf{sparsity} and \textbf{fill-in} \\
      \qquad perform \textbf{minimal algebraic manipulations} \\
      \qquad allow for custom \textbf{pivoting strategies}
    \end{small}
  \item<3->However\dots
    \begin{enumerate}
      \normalsize
      \item output's numerical stability is not guaranteed \\
      \begin{small}
        \qquad \textbf{How to ensure numerical stability?}
      \end{small}
      \item expressions grow unpredictably during manipulation \\
      \begin{small}
        \qquad \textbf{How to manage large symbolic expressions?}
      \end{small}
    \end{enumerate}
  \end{itemize}
\end{frame}

\begin{frame}{Symbolic Matrix Factorization}{Symbolic Pivoting Strategy}
  \vspace{-1.5em}
  \hic{How to ensure numerical stability?}
  \begin{columns}
    \begin{column}[c]{0.47\textwidth}
      A good symbolic \textbf{pivoting strategy} \dots
      \begin{enumerate}
        \item<1> selects the \textbf{minimum-degree} entry \\
        \begin{small}
          \qquad Limit fill-in
        \end{small}
        \item<2> if numeric, selects the \textbf{biggest} entry \\
        \begin{small}
          \qquad Improve numerical stability
        \end{small}
        \item<3> selects \textbf{simplest} and \textbf{min-degree} entry \\
        \begin{small}
          \qquad Limit expression swell
        \end{small}
        \item<4> looks for the \textbf{best choice} \\
        \begin{small}
          \qquad Full-pivoting strategy
        \end{small}
      \end{enumerate}
    \end{column}
    \begin{column}[c]{0.53\textwidth}
      \begin{algorithmic}\scriptsize
        \Procedure{SymbolicPivoting}{$\m{A}$, $k$}
        \State \only<1>{\hlc}{$\m{d}^r, \, \m{d}^c \gets \text{ComputeDegrees}(\m{A})$}
          \For{\only<4>{\hlc}{$i$ \textbf{from} $k$ \textbf{to} $m$}}
          \For{\only<4>{\hlc}{$j$ \textbf{from} $k$ \textbf{to} $n$}}
              \State $D_{ij} \gets \infty$
              \IfThen{$A_{ij} \neq 0$}
              {\only<1>{\hlc}{$D_{ij} \gets d^r_{i} \, \max(0, \, d^c_j-1) + d^c_j \, \max(0, \, d^r_i-1)$}}
            \EndFor
          \EndFor
          \State \only<1>{\hlc}{$\mathcal{P} \gets \text{Sort}(\m{D})$}
          \State $q, \, l \gets \, 0, \, 0$  and $p, \, p_c, \, p_n \gets \infty, \, \infty, \, \infty$
          \For{\textbf{all} $(i, j)$ \textbf{in} $\mathcal{P}$}
            \IfThen{$p_c \neq \infty$ \textbf{and} $D_{ij} > D_{ql}$}{\textbf{break}}
            \State $t \gets A_{ij}$
            \IfThen{$\text{Signature}(t) = 0$}{\textbf{continue}}
            \State $t \gets \text{Simplify}(t)$ and \only<3>{\hlc}{$t_c \gets \text{ExpressionComplexity}(t)$} and $t_n \gets \infty$
            \IfThen{$t$ is numeric}{\only<2>{\hlc}{$t_n \gets \max(1, \, \text{abs}(t))$}}
            \If{\only<3>{\hlc}{$t_c < p_c$} \textbf{or} (\only<2>{\hlc}{$t_c = p_c$} \textbf{and} \only<2>{\hlc}{$t_n > p_n$})}
              \State $q, \, l \gets i, \, j$ and $p, \, p_c, \, p_n \gets t, \, t_c, \, t_n$
            \EndIf
          \EndFor
          \State \textbf{return} $p, \, q, \, l$
        \EndProcedure
      \end{algorithmic}
    \end{column}
  \end{columns}
\end{frame}

\begin{frame}{Symbolic Matrix Factorization}{\acs{LEM} During Factorization}
  \vspace{-1.5em}
  \hic{How to manage large symbolic expressions?}
  \begin{columns}
    \begin{column}[c]{0.32\textwidth}
      Veils are introduced during \dots
      \begin{itemize}
        \item<1> \textbf{Factorization step} \\
        \begin{small}
          \qquad gaussian elimination \\
          \qquad Schur complement
        \end{small}
        \item<2> \textbf{Solution step} \\
        \begin{small}
          \qquad forward substitution \\
          \qquad backward substitution
        \end{small}
      \end{itemize}
    \end{column}
    \begin{column}[c]{0.36\textwidth}
      \uncover<1>{\begin{algorithmic}\scriptsize
        \Procedure{LU}{$\m{A}, \, k$}
          \State $\m{M} \gets \m{A}$
          \State $rnk \gets \min(m, \, n)$
          \For{$k$ \textbf{from} $1$ \textbf{to} $rnk$}
            \State $p, \, q, \, l \gets \text{SymbolicPivoting}(\m{M}, \, k)$
            \If{$p = 0$}
              \State $rnk \gets k-1$ and \textbf{break}
            \EndIf
            \State $\m{r}_k, \, \m{c}_k \gets \, q, \, l$
            \State $\m{M} \gets \text{SwapRowsCols}(\m{M}, \, k, \, q,  \, l)$
            \For{$i$ \textbf{from} $k+1$ \textbf{to} $m$}
              \State \only<1>{\hlc}{$M_{kk} \gets \text{Veil}(M_{kk})$}
              \State \only<1>{\hlc}{$M_{ik} \gets \text{Veil}(\text{Normalizer}(M_{ik}/M_{kk}))$}
              \For{$j$ \textbf{from} $k+1$ \textbf{to} $n$}
                \State \only<1>{\hlc}{$M_{ij} \gets \text{Veil}(\text{Normalizer}(M_{ij} - M_{ik}M_{kj}))$}
              \EndFor
            \EndFor
          \EndFor
          \State $\m{P}, \, \m{Q} \gets \text{PermutationMatrices}(\m{r}, \, \m{c})$
          \State $\m{L}, \, \m{U} \gets \text{LowerUpperMatrices}(\m{M})$
          \State \textbf{return} $\m{L}, \, \m{U}, \, \m{P}, \, \m{Q}, \, \m{r}, \, \m{c}, \, rnk$
        \EndProcedure
      \end{algorithmic}}
    \end{column}
    \begin{column}[c]{0.32\textwidth}
      \uncover<2>{\begin{algorithmic}\scriptsize
        \Procedure{SolveLU}{$\m{L}, \, \m{U}, \, \m{P}, \, \m{Q}, \, \m{b}$}
          \State $\m{y} \gets \m{P}\m{b}$
          \State $m, \, n \gets \text{Size}(\m{L})$
          \For{$i$ \textbf{from} $2$ \textbf{to} $m$}
            \State \only<2>{\hlc}{$y_i \gets \text{Veil}(y_i - \sum\nolimits_{j=1}^{i-1} L_{ij}y_j)$}
          \EndFor
          \State $x_n \gets \text{Veil}(y_n/U_{nn})$
          \For{$i$ \textbf{from} $n-1$ \textbf{to} $1$}
            \State \only<2>{\hlc}{$x_i \gets \text{Veil}(y_i - {\sum\nolimits_{j=i+1}^{n}} U_{ij}x_j)$}
            \State \only<2>{\hlc}{$x_i \gets \text{Veil}(x_i/U_{ii})$}
          \EndFor
          \State $\m{x} \gets \m{Q}^\top\m{x}$
        \EndProcedure
      \end{algorithmic}}
    \end{column}
  \end{columns}
\end{frame}

% That's all Folks!
